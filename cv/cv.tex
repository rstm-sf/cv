%%%%%%%%%%%%%%%%%
% This is an example CV created using altacv.cls (v1.3, 10 May 2020) written by
% LianTze Lim (liantze@gmail.com), based on the
% Cv created by BusinessInsider at http://www.businessinsider.my/a-sample-resume-for-marissa-mayer-2016-7/?r=US&IR=T
%
%% It may be distributed and/or modified under the
%% conditions of the LaTeX Project Public License, either version 1.3
%% of this license or (at your option) any later version.
%% The latest version of this license is in
%%    http://www.latex-project.org/lppl.txt
%% and version 1.3 or later is part of all distributions of LaTeX
%% version 2003/12/01 or later.
%%%%%%%%%%%%%%%%

%% If you are using \orcid or academicons
%% icons, make sure you have the academicons
%% option here, and compile with XeLaTeX
%% or LuaLaTeX.
% \documentclass[10pt,a4paper,academicons]{altacv}

%% Use the "normalphoto" option if you want a normal photo instead of cropped to a circle
% \documentclass[10pt,a4paper,normalphoto]{altacv}

\documentclass[10pt,a4paper,ragged2e,withhyper]{altacv}

%% AltaCV uses the fontawesome5 and academicon fonts
%% and packages.
%% See http://texdoc.net/pkg/fontawesome5 and http://texdoc.net/pkg/academicons for full list of symbols. You MUST compile with XeLaTeX or LuaLaTeX if you want to use academicons.

% Change the page layout if you need to
\geometry{left=1.25cm,right=1.25cm,top=1.5cm,bottom=1.5cm,columnsep=1.2cm}

% The paracol package lets you typeset columns of text in parallel
\usepackage{paracol}


% Change the font if you want to, depending on whether
% you're using pdflatex or xelatex/lualatex
\ifxetexorluatex
  % If using xelatex or lualatex:
  \setmainfont{Lato}
\else
  % If using pdflatex:
  \usepackage[default]{lato}
\fi

\definecolor{VividPurple}{HTML}{3E0097}
\definecolor{SlateGrey}{HTML}{2E2E2E}
\definecolor{LightGrey}{HTML}{666666}
\colorlet{tagline}{VividPurple}
\colorlet{heading}{VividPurple}
\colorlet{headingrule}{VividPurple}
\colorlet{accent}{VividPurple}
\colorlet{emphasis}{SlateGrey}
\colorlet{body}{LightGrey}

% Change the bullets for itemize and rating marker
% for \cvskill if you want to
\renewcommand{\itemmarker}{{\small\textbullet}}
\renewcommand{\ratingmarker}{\faCircle}

\NewInfoField{bitbucket}{\faBitbucket}[https://bitbucket.org/]

\addbibresource{references.bib}

\begin{document}

\name{Rustam Sayfutdinov}
\tagline{.NET Software Engineer}
\photoR{2.5cm}{logo}
\personalinfo{
  \email{rstm.sf@gmail.com}
  \github{rstm-sf}
  \bitbucket{rstm-sf}
}

\makecvheader

%% Depending on your tastes, you may want to make fonts of itemize environments slightly smaller
\AtBeginEnvironment{itemize}{\small}

%% Set the left/right column width ratio to 6:4.
\columnratio{0.6}

% Start a 2-column paracol. Both the left and right columns will automatically
% break across pages if things get too long.
\begin{paracol}{2}

\cvsection{Experience}

\cvevent{Backend .NET Developer}
{CraftTalk}
{October 2021 -- Present}
{Remote}

Software development for contact centers:

\begin{itemize}

  \item Web API;
  \smallskip

  \item extending of possibilities of a platform;
  \smallskip

  \item small edit of the frontend for changes.

\end{itemize}

\divider

\cvevent{Full Stack .NET Developer}
{Junior Projects}
{June 2021 -- September 2021}
{Remote}

Software development for education:

\begin{itemize}

  \item user interface;
  \smallskip

  \item automation of reports;
  \smallskip

  \item integration with various services.

\end{itemize}

\divider

\cvevent{С\# Developer}
{Actual Technology Russia}
{January 2019 -- March 2021}
{Kazan, Russia}

Development of software for geological field analysis and planning of geological and technical measures:

\begin{itemize}

  \item user interface;
  \smallskip

  \item automation of reports;
  \smallskip

  \item application integration with Apache~Hive;
  \smallskip

  \item writing .NET-wrappers for C++ libraries;
  \smallskip

  \item setting up auto-build of projects.

\end{itemize}

\divider

\cvevent{Engineer}
{The Computer Science Research Institute of USATU}
{Jule 2016 -- August 2017}
{Ufa, Russia}

Investigation of the applicability of NVIDIA GPU for solving SLAEs with a dense matrix in modeling.

\divider

\cvevent{Laboratory Assistant}
{The Computer Science Research Institute of USATU}
{May 2016 -- June 2017}
{Ufa, Russia}

Optimization and profiling software:

\begin{itemize}

  \item the influence of the numbering of the computational grid on the performance in modeling;
  \smallskip

  \item the impact of using intel-intrinsics for matrix-vector operations;
  \smallskip

  \item scalability of MPI-application of modeling on the computational cluster of USATU.

\end{itemize}

%% Switch to the right column. This will now automatically move to the second
%% page if the content is too long.
\switchcolumn

\cvsection{Strengths}

\cvtag{.NET}
\cvtag{ASP.NET Core}
\cvtag{Razor Pages}
\cvtag{Winforms}
\cvtag{C\#}
\cvtag{F\#}
\cvtag{C/C++}

\cvsection{Languages}

\cvskill{Russian}{5}
\smallskip
\cvskill{English}{1}

\cvsection{Education}

\cvevent{Master's degree
\newline ``Fundamental CS and IT''}
{ICMIT, KFU}
{2017 -- 2019}
{Kazan, Russia}

\divider

\cvevent{Online Course ``Machine Learning''}
{Stanford University}
{2018}
{Coursera}

\divider

\cvevent{Bachelor's degree
\newline ``Applied Mathematics and CS''}
{FGS, USATU}
{2012 -- 2017}
{Ufa, Russia}

\cvsection{Publications}

\nocite{*}

\printbibliography[heading=pubtype,title={\printinfo{\faUsers}{Conference Proceedings}},type=inproceedings]

\end{paracol}

\end{document}

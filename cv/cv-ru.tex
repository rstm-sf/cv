\documentclass[a4paper,12pt]{article}

\usepackage[margin=2cm]{geometry}
\usepackage{titlesec}
\usepackage{marvosym}
\usepackage{xcolor}
\usepackage{enumitem}
\usepackage[hidelinks]{hyperref}
\usepackage{fancyhdr}
\usepackage{tabularx}


\usepackage{ifxetex,ifluatex}
\newif\ifxetexorluatex
\ifxetex
  \xetexorluatextrue
\else
  \ifluatex
    \xetexorluatextrue
  \else
    \xetexorluatexfalse
  \fi
\fi

\ifxetexorluatex
  \usepackage{fontspec}
\else
    \usepackage[english,russian]{babel}
    \usepackage{cmap}
    \usepackage[utf8]{inputenc}
    \usepackage[T1]{fontenc}
    \input{glyphtounicode}
\fi

% Change the font if you want to, depending on whether
% you're using pdflatex or xelatex/lualatex
\ifxetexorluatex
  % If using xelatex or lualatex:
  \setmainfont{Lato}
\else
  % If using pdflatex:
  \usepackage[default]{lato}
\fi


\pagestyle{fancy}
\fancyhf{} % clear all header and footer fields
\fancyfoot{}
\renewcommand{\headrulewidth}{0pt}
\renewcommand{\footrulewidth}{0pt}

% Adjust margins
\addtolength{\oddsidemargin}{-0.5in}
\addtolength{\evensidemargin}{-0.5in}
\addtolength{\textwidth}{1in}
\addtolength{\topmargin}{-.5in}
\addtolength{\textheight}{1.0in}

\urlstyle{same}

\raggedbottom
\raggedright
\setlength{\footskip}{4.35004pt}

% Sections formatting
\titleformat{\section}{
  \vspace{-4pt}\scshape\raggedright\large
}{}{0em}{}[\color{black}\titlerule \vspace{-5pt}]

\ifxetexorluatex
\else
  % Ensure that generate pdf is machine readable/ATS parsable
  \pdfgentounicode=1
\fi

%-------------------------
% Custom commands
\newcommand{\resumeItem}[1]{
  \item\small{
    {#1 \vspace{-2pt}}
  }
}

\newcommand{\resumeSubheading}[4]{
  \vspace{-2pt}\item
    \begin{tabular*}{0.97\textwidth}[t]{l@{\extracolsep{\fill}}r}
      \textbf{#1} & #2 \\
      \textit{\small#3} & \textit{\small #4} \\
    \end{tabular*}\vspace{-7pt}
}

\newcommand{\resumeSubSubheading}[2]{
  \item
  \begin{tabular*}{0.97\textwidth}{l@{\extracolsep{\fill}}r}
    \textit{\small#1} & \textit{\small #2} \\
  \end{tabular*}\vspace{-7pt}
}

\newcommand{\resumeProjectHeading}[2]{
  \item
  \begin{tabular*}{0.97\textwidth}{l@{\extracolsep{\fill}}r}
    \small#1 & #2 \\
  \end{tabular*}\vspace{-7pt}
}

\newcommand{\resumeSubItem}[1]{\resumeItem{#1}\vspace{-4pt}}

\renewcommand\labelitemii{$\vcenter{\hbox{\tiny$\bullet$}}$}

\newcommand{\resumeSubHeadingListStart}{\begin{itemize}[leftmargin=0.15in, label={}]}
\newcommand{\resumeSubHeadingListEnd}{\end{itemize}\vspace{-3pt}}
\newcommand{\resumeItemListStart}{\begin{itemize}}
\newcommand{\resumeItemListEnd}{\end{itemize}\vspace{-5pt}}

\newcommand{\resumeSkillListStart}{\begin{itemize}[leftmargin=0.20in, label={}]}
\newcommand{\resumeSkillListEnd}{\end{itemize}\vspace{-3pt}}

\newcommand{\resumeSkillItem}[2]{
  \item\small{
    \textbf{#1}{: #2}\vspace{-6pt}
  }
}


\usepackage[backend=biber,style=authoryear,sorting=ydnt]{biblatex}
\addbibresource{references.bib}
\defbibheading{pubtype}{\hspace{18pt}\vspace{-2pt}{\bfseries{#1}}\medskip}
\renewcommand{\bibsetup}{\vspace{-6pt}}
\renewcommand*{\bibfont}{\small}
\AtEveryCitekey{item[]}

%-------------------------------------------
%%%%%%  RESUME STARTS HERE  %%%%%%%%%%%%%%%%%%%%%%%%%%%%


\begin{document}


\begin{center}
  \textbf{\Huge \scshape Рустам Сайфутдинов} \\ \vspace{1pt}
  \small
  \href{mailto:rstm.sf@gmail.com}{\underline{rstm.sf@gmail.com}} $|$
  \href{https://github.com/rstm-sf}{\underline{github.com/rstm-sf}}
\end{center}


%-----------EXPERIENCE-----------
\section{Опыт}
  \resumeSubHeadingListStart

    \resumeSubheading
      {Backend .NET Developer}{Окт. 2021 -- По наст. вр.}
      {CraftTalk}{}
      \resumeItemListStart
        \resumeItem{Разработка ПО для контакт-центров, включая расширение функционала базы знаний, веб-чата и интеграций.}
        \resumeItem{Оптимизация работы Redis и Elasticsearch, внедрение OpenSearch в качестве альтернативы. Решение проблем межсервисного взаимодействия.}
        \resumeItem{Технологии: F\#, C\#, ASP.NET Core, RabbitMQ, S3, ReactTS, AngularJs, NodeJs, Python, Go, Docker, OpenTelemetry, Grafana, Loki, Tempo, Prometheus, YouTrack, GitLab.}
      \resumeItemListEnd

    \resumeSubheading
      {Full Stack .NET Developer}{Июнь 2021 -- Сент. 2021}
      {Junior Projects}{}
      \resumeItemListStart
        \resumeItem{Разработка ПО для образовательных проектов с акцентом на расширение интерфейса администратора.}
        \resumeItem{Интеграция внешних сервисов и автоматизация отчетности.}
        \resumeItem{Технологии: C\#, ASP.NET Core, Razor, EF Core, Postgres, Jira, New Relic.}
      \resumeItemListEnd

    \resumeSubheading
      {С\# Developer}{Янв. 2019 -- Март 2021}
      {Актуальные технологии}{Казань, Россия}
      \resumeItemListStart
        \resumeItem{Разработка ПО для геолого-промыслового анализа и планирования мероприятий, автоматизация создания отчетов.}
        \resumeItem{Исследование и внедрение интеграции с Apache Hive.}
        \resumeItem{Технологии: C\#, WinForms, C/C++, Postgres, YouTrack, TeamCity.}
      \resumeItemListEnd

    \resumeSubheading
      {Лаборант}{Май 2016 -- Авг. 2017}
      {Институт компьютерных исследований при НИЧ УГАТУ}{Уфа, Россия}
      \resumeItemListStart
         \resumeItem{Оптимизация производительности ПО для моделирования физических процессов.}
         \resumeItem{Технологии: C/C++, CUDA, Intel Intrinsics, MPI, Redmine.}
      \resumeItemListEnd

  \resumeSubHeadingListEnd


%-----------PROJECTS-----------
\section{Проекты}
    \resumeSubHeadingListStart
      \resumeProjectHeading
          {\textbf{view-avalonia-preview-vscode} $|$ \emph{.NET, TypeScript, HTML}}{Февр. 2021 -- Май 2021}
          \resumeItemListStart
            \resumeItem{Разработка интеграции превьювера в VS Code}
            \resumeItem{Разработка MSBuild-таски получения настроек .NET проекта на Avalonia для дальнейшего передачи параметров запуска превьювера}
          \resumeItemListEnd
    \resumeSubHeadingListEnd


%-----------EDUCATION-----------
\section{Образование}
  \resumeSubHeadingListStart
    \resumeSubheading
      {ИВМиИТ, К(П)ФУ}{Казань, Россия}
      {Магистр по направлению <<Фундаментальная информатика и информационные технологии>>}{2017 -- 2019}
    \resumeSubheading
      {ОНФ, УГАТУ}{Уфа, Россия}
      {Бакалавр по направлению <<Прикладная математика и информатика>>}{2012 -- 2017}
  \resumeSubHeadingListEnd

%
%-----------PUBLICATIONS-----------
\section{Публикации}
\nocite{*}
\printbibliography[heading=pubtype,title={Труды конференции},type=inproceedings]


%-------------------------------------------
\end{document}

%%%%%%%%%%%%%%%%%
% This is an example CV created using altacv.cls (v1.3, 10 May 2020) written by
% LianTze Lim (liantze@gmail.com), based on the
% Cv created by BusinessInsider at http://www.businessinsider.my/a-sample-resume-for-marissa-mayer-2016-7/?r=US&IR=T
%
%% It may be distributed and/or modified under the
%% conditions of the LaTeX Project Public License, either version 1.3
%% of this license or (at your option) any later version.
%% The latest version of this license is in
%%    http://www.latex-project.org/lppl.txt
%% and version 1.3 or later is part of all distributions of LaTeX
%% version 2003/12/01 or later.
%%%%%%%%%%%%%%%%

%% If you are using \orcid or academicons
%% icons, make sure you have the academicons
%% option here, and compile with XeLaTeX
%% or LuaLaTeX.
% \documentclass[10pt,a4paper,academicons]{altacv}

%% Use the "normalphoto" option if you want a normal photo instead of cropped to a circle
% \documentclass[10pt,a4paper,normalphoto]{altacv}

\documentclass[10pt,a4paper,ragged2e,withhyper]{altacv}

%% AltaCV uses the fontawesome5 and academicon fonts
%% and packages.
%% See http://texdoc.net/pkg/fontawesome5 and http://texdoc.net/pkg/academicons for full list of symbols. You MUST compile with XeLaTeX or LuaLaTeX if you want to use academicons.

% Change the page layout if you need to
\geometry{left=1.25cm,right=1.25cm,top=1.5cm,bottom=1.5cm,columnsep=1.2cm}

% The paracol package lets you typeset columns of text in parallel
\usepackage{paracol}


% Change the font if you want to, depending on whether
% you're using pdflatex or xelatex/lualatex
\ifxetexorluatex
  % If using xelatex or lualatex:
  \setmainfont{Lato}
\else
  % If using pdflatex:
  \usepackage[default]{lato}
\fi

\definecolor{VividPurple}{HTML}{3E0097}
\definecolor{SlateGrey}{HTML}{2E2E2E}
\definecolor{LightGrey}{HTML}{666666}
\colorlet{tagline}{VividPurple}
\colorlet{heading}{VividPurple}
\colorlet{headingrule}{VividPurple}
\colorlet{accent}{VividPurple}
\colorlet{emphasis}{SlateGrey}
\colorlet{body}{LightGrey}

% Change the bullets for itemize and rating marker
% for \cvskill if you want to
\renewcommand{\itemmarker}{{\small\textbullet}}
\renewcommand{\ratingmarker}{\faCircle}

\NewInfoField{bitbucket}{\faBitbucket}[https://bitbucket.org/]

\addbibresource{references.bib}

\begin{document}

\name{Рустам Сайфутдинов}
\tagline{.NET разработчик}
\photoR{2.5cm}{logo}
\personalinfo{
  \email{rstm.sf@gmail.com}
  \github{rstm-sf}
  \bitbucket{rstm-sf}
}

\makecvheader

%% Depending on your tastes, you may want to make fonts of itemize environments slightly smaller
\AtBeginEnvironment{itemize}{\small}

%% Set the left/right column width ratio to 6:4.
\columnratio{0.6}

% Start a 2-column paracol. Both the left and right columns will automatically
% break across pages if things get too long.
\begin{paracol}{2}

\cvsection{Опыт}

\cvsubsection{Работа}

\cvevent{С\# Developer}
{Актуальные технологии}
{Январь 2019 -- Март 2021}
{Казань, Россия}

\begin{itemize}

  \item разработка desktop-приложений, используя MVP-паттерн и IoC-контейнер
  \smallskip

  \item применение ADO.NET для Postgres и Apache~Hive
  \smallskip

  \item применение P/Invoke для вызова C/C++ кода
  \smallskip

  \item настройка автосборок .NET проектов на TeamCity

\end{itemize}

\divider

\cvevent{Лаборант}
{Институт компьютерных исследований при НИЧ УГАТУ}
{Май 2016 -- Август 2017}
{Уфа, Россия}

\begin{itemize}

  \item исследование применимости ГПУ NVIDIA для
  решения СЛАУ с плотной матрицей при гидродинамическом моделировании
  \smallskip

  \item исследование влияния нумерации расчетной сетки на
  \newline производительность при гидродинамическом моделировании
  \smallskip

  \item исследование влияния использования intel-intrinsics при матрично-векторных операциях в специализированном решателе
  \smallskip

  \item профилирование на масштабируемость  MPI-приложения
  \newline гидродинамического моделирования на выч.\,кластере УГАТУ

\end{itemize}

{\color{headingrule}\rule{\linewidth}{1pt}\par}\medskip

\cvsubsection{Open Source}

\smallskip

\begin{itemize}

  \item клиент-серверное CRUD
  \href{https://github.com/rstm-sf/ClientServerCRUD}{приложение},
  с использованием Elm-style SPA и EFCore.FSharp
  \smallskip

  \item бекенд для SPA, с использованием
  \href{https://github.com/rstm-sf/ASP.NETCoreBackendFor-Angular-GettingStarted}{ASP.NET Core} или
  \href{https://github.com/rstm-sf/GiraffeBackendFor-Angular-GettingStarted}{Giraffe}
  \smallskip

  \item браузерное приложение
  \href{https://github.com/rstm-sf/tixy.Fable}{tixy.Fable},
  использующее Fable-компилятор для преобразования в JS
  \smallskip

  \item возможность указать
  \href{https://github.com/oxyplot/oxyplot/pull/1736}{дистанцию}
  для трекера кривой в библиотеки OxyPlot
  \smallskip

  \item поддержка
  \href{https://github.com/AvaloniaUI/Avalonia/pull/5415}{сборки}
  Avalonia в .NET 5 SDK
  \smallskip

  \item поддержка .NET 5 в
  \href{https://github.com/AvaloniaUI/AvaloniaVS/pull/181}{VS-превьювере}
  для Avalonia XAML 
  \smallskip

  \item обработка событий мыши в
  \href{https://github.com/AvaloniaUI/Avalonia/pull/4418}{web-превьювере}
  и \href{https://github.com/ForNeVeR/AvaloniaRider/pull/82}{Rider-превьювере}
  для Avalonia XAML 
  \smallskip

  \item разработка кроссплатформенных desktop-приложений, используя MVVM-паттерн:
  \href{https://github.com/rstm-sf/RunSQL}{RunSQL} и
  \href{https://github.com/rstm-sf/TrajectoryFinder2D}{TrajectoryFinder2D}
  \smallskip

  \item исправление ошибок в проектах:
  \href{https://github.com/AvaloniaUI/avaloniaui.net/commits?author=rstm-sf}{avaloniaui.net} и
  \href{https://github.com/ForNeVeR/wpf-math/commits?author=rstm-sf}{wpf-math}
  \smallskip

  \item консолидация .NET кодировок в проекте 
  \href{https://github.com/CharsetDetector/UTF-unknown/commits?author=rstm-sf}{UTF-unknown}
  \smallskip

  \item исследование моделей использования SQL и NoSQL БД с
  \href{https://bitbucket.org/rstm-sf/game_with_db}{примерами} в
  PostgreSQL, MongoDB, MonetDB, Riak, Redis и ArangoDB

\end{itemize}

%% Switch to the right column. This will now automatically move to the second
%% page if the content is too long.
\switchcolumn

\cvsection{Навыки}

\cvtag{.NET}
\cvtag{C\#}

\cvsection{Языки}

\cvskill{Русский}{5}
\smallskip
\cvskill{Английский}{1}

\cvsection{Образование}

\cvevent{Магистр по направлению
\newline \mbox{<<Фундаментальная информатика и}
\newline \mbox{информационные технологии>>}}
{ИВМиИТ, К(П)ФУ}
{2017 -- 2019}
{Казань, Россия}

\divider

\cvevent{Онлайн-курс <<Machine Learning>>}
{Stanford University}
{2018}
{Coursera}

\divider

\cvevent{Бакалавр по направлению
\newline \mbox{<<Прикладная математика и}
\newline \mbox{информатика>>}}
{Общенаучный факультет, УГАТУ}
{2012 -- 2017}
{Уфа, Россия}

\cvsection{Публикации}

\nocite{*}

\printbibliography[heading=pubtype,title={\printinfo{\faUsers}{Труды конференции}},type=inproceedings]

\end{paracol}

\end{document}
